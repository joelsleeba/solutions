% TeX_root = ../main.tex

\chapter{Fourier Series on the Circle}

\section{Motivation and Heuristics}

\subsection{Motivations from Physics}

\subsubsection{The Vibrating String}

\begin{exercise}
  Use Induction. Show that $\sin^3(x)$ can be written as a sum of three sines.
\end{exercise}

\begin{exercise}
  Use contradiction. Integrate over $[0, \pi]$.
\end{exercise}

\begin{exercise}
  Use the same logic as above.
\end{exercise}

\begin{exercise}
  Easy
\end{exercise}

\subsubsection{The Heat Flow in Solids}

\subsection{Absolutely Convergent Trigonometric Series}

\setcounter{theorem}{6}

\begin{exercise}
  Use triangle inequality on the definition of limit. Then use the
  given hint and the absolute convergence of the sequence again. Now
  use dominated convergence theorem for the counting measure.
\end{exercise}

\begin{exercise}
  Same method as above the exercise using induction.
\end{exercise}

\begin{corollary}
  Why is
  \begin{align*}
    \int_{\mathbb{T}}  g(\phi - \theta) e^{-i n \theta} \ d \theta  =
    \int_{\mathbb{T}}  g(y) e^{in(y - \phi)} \ d  y
  \end{align*}
  Shouldn't that be the negative of the RHS.
  % \begin{align*}
  %   \int_{\mathbb{T}}  g(\phi - \theta) e^{-i n \theta} \ d \theta  =
  %   -\int_{\mathbb{T}}  g(y) e^{-in(\phi - y)} \ d  y = -e^{-in
  %   \phi} \int_{\mathbb{T}}  g(y) e^{iny} \ d y
  % \end{align*}
  % to get that the $n$-th Fourier coefficient for $g(\phi- \theta)$
  % is $-e^{-in\phi} D_{-n}$ where $D_{-n}$ is the $-n$-th Fourier
  % coefficient of $g(x)$. Now use
\end{corollary}
%
