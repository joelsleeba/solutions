% TeX_root = ../main.tex

\chapter{Measurable Functions}

\begin{exercise}[I]
	Give an example of a function $f$ on $X$ to $\mathbb{R}$ which is not $X$ measurable but $f^2$ and $|f|$ are.
\end{exercise}
\begin{solution}
	Let $X = \mathbb{R}$ with the $\sigma$-algebra $\mathcal{X} = \{ \emptyset, \mathbb{R}, \{ 1 \}, \mathbb{R} \setminus \{ 1 \} \}$. Now let $f$ be defined as \[
		f(x) = \begin{cases}
			1, \quad x \in (0, \infty) \\
			-1, \quad \textrm{otherwise}
		\end{cases}
	\]
	Then $f^2 = |f| =$ \textbf{1} the constant function is measurable, but $f^{-1}((-2, -1)) = \{ -1 \} \notin \mathcal{X}$. Hence $f$ is not measurable.
\end{solution}

\begin{exercise}[S]
	Let $f$ be a complex valued function defined on a measurable space $(X, \mathcal{X})$. Show that $f$ is $X$-measurable if and only if \[
		E_{a, b, c, d} := \{ x \in X \ : \ a < \Re(f)(x) < b, c < \Im(f)(x) < d \} \in \mathcal{X}\] for all real numbers $a, b, c, d$.

	More generally show that $f$ is $X$-measurable if and only if $f^{-1}(G) \in \mathcal{X}$ for every open set $G$ in the complex plane.
\end{exercise}
\begin{solution}
	Let $f = f_1 + if_2$. Then $f$ is $X$-measurable if and only if both $f_1$ and $f_2$ are $X$-measurable. Hence we get $E_{a, b, c, d} = f_1^{-1}(a, b) \cap f_2^{-1}(c, d) \in \mathcal{X}$.

	Conversely if $E_{a, b, c, d} \in \mathcal{X}$ for each real numbers $a, b, c, d$, then we can show that $f_1, f_2$ are $X$-measurable by working with $E_{a, b, -\infty, \infty}$ and $E_{-\infty, \infty, c, d}$. (Note that here we'll violate the requirement that $a, b, c, d$ are reals, but we can modify this to work with reals by choosing large enough values instead of infinities).

	Now for the general case, note that $\mathbb{R}^2$(product topology) and $\mathbb{C}$ are topologically equivalent. Since arbitrary open set $U \in \mathbb{R}$ can be written as a countable union of disjoint open intervals, an open set $G \in \mathbb{R}^2$ can be written as countable union of disjoint open rectangles. Now the above result using $E_{a, b, c, d}$ gives the equivalence.
\end{solution}

\begin{exercise}[T]
	Show that sums, products, and limits of complex-valued measurable functions are measurable.
\end{exercise}
\begin{solution}
	Use the algebra of real measurable functions.
\end{solution}

\begin{exercise}[U]
	Show that a function $f: X \to \mathbb{R}$(or to $\bar{\mathbb{R}}$) is $X$-measurable if and only if the collection $A_\alpha = f^{-1}(\alpha, \infty) \in \mathcal{X}$ for each $\alpha \in \mathbb{Q}$. Equivalently $B_\alpha = f^{-1}(-\infty, \alpha) \in  \mathcal{X}$
\end{exercise}
\begin{solution}
	Follows from the fact that the collection $(\alpha, \infty)$ and $(-\infty , \alpha)$ where $\alpha \in \mathbb{Q}$ independently generate the Borel sigma algebra for $\mathbb{R}$.
\end{solution}

\begin{exercise}[V]
	See the definition for monotone classes from the problem. Show that for any nonempty collection of subsets of $X$, there is a smallest monotone class containing $A$.
\end{exercise}
\begin{solution}
	Clearly the $\sigma$-algebra generated by $A$ is a monotone class containing $A$. Now take the intersection of all the monotone classes containing $A$. This is a monotone class. (Verify).
\end{solution}
