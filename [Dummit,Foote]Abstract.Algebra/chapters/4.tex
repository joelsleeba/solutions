% TeX_root = ../main.tex

\setcounter{chapter}{3}
\chapter{Group Actions}

\section{Group Actions and Permutation Representations}

\setcounter{proposition}{1}
\begin{proposition}
  If $G$ is a group acting on a set $A \ni a$. Then the cardinality
  of the orbit of $a$,  $|O_a| = |G:G_a|$.
\end{proposition}
\begin{proof}
  Notice that every element of $G_a$ acts trivially on $a$. Therefore
  $h(g(a)) = h(a)$ for all $g \in G_a$. Hence the coset $hG_a$ acts
  on $a$ and gives $h(a)$. Hence we see that $|O_a| \le |G:G_a|$.

  Now conversely if $hG_a$ and $kG_a$ acts on $a$ to give the same
  output, i.e $h(a) = k(a)$, then $h^{-1}k(a) = a$. Therefore
  $h^{-1}k \in G_a$ and hence $hG_a = kG_a$. Hence we get $|O_a| = |G:G_a|$.
\end{proof}

\begin{questions}
  \question [3]
  \begin{solution}
    Show that if $\sigma(a) = a$ for some $\sigma \in G$, then
    $\sigma \in G_x$ for all $x \in A$ by the Abelianess. This
    contradicts $G \leqslant S_A$.

    Now for the other one use the above proposition along with $G_a = \{ e \}$.
  \end{solution}
\end{questions}


